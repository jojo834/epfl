\documentclass[14pt]{extreport}
\usepackage{graphicx}

%\input{preamble}
%\input{macros}
%\input{letterfonts}

\title{\Huge{Some Class}\\Random Examples}
\author{\huge{Your Name}}
\date{}

\begin{document}

\section*{Dessin}
\includegraphics[width=\textwidth]{test_xournalpp.pdf}


\subsection*{Paramètres}


\subsection*{Variables}


\subsection*{Fonctions connues}


\subsection*{Fonctions à determiner}


\subsection*{Objectifs}


\section*{Référentiel/Observateur}


\section*{Forces}

\subsection*{Forces intérieures}
Les forces intérieures sont exercées entre des objets intérieurs
au système. Par la loi de l'action et de la réaction, on a:
\[ \vec{F}_{A \rightarrow B} = -\vec{F}_{B \rightarrow A}
\Longleftrightarrow \vec{F}_{A \rightarrow B} +
\vec{F}_{B \rightarrow A} = 0 \]
Par conséquence:
\[ \sum \vec{F}_{\mathrm{int}} = 0 \]

\subsection*{Forces extérieures}


\subsubsection*{Énumération des forces extérieures}


\subsubsection*{Dessin des forces extérieures}
\includegraphics[width=\textwidth]{test_xournalpp.pdf}


\section*{Lois appliquables}


\section*{Coordonnées}


\section*{Equations du mouvement}


\section*{Vérifications}



\end{document}
